\section{Reduction: 3SAT to Subset sum}
In this chapter we try to prove that the \textbf{Subset sum} problem is \textbf{NP-Complete} by providing a polynomial many-to-one reduction from \textbf{3SAT} to \textbf{Subset sum}, knowing that \textbf{3SAT} belongs to \textbf{NP-complete}.
\subsection{Definitions}
\subsubsection{3-satisfiability}
Given a list of boolean variables $U=\{u_1, u_2, ..., u_n\}$ and a set of clauses,
\newline
$G=\{c_1, c_2, ..., c_m\}$ in 3-Conjunctive normal form (3-CNF), find an assignment of truth values for $U$ such that all clauses in $G$ are satisfied.
\newline
\newline
\textbf{Example}:
\newline
\textbf{Input}: \newline
$U=\{x_1, x_2, x_3, x_4\}$\newline
$G=\{c_1, c_2, c_3\}$\newline
Where:\newline
$c_1 = x_1 \vee x_2 \vee \overline{x_3}$\newline
$c_2 = \overline{x_2} \vee x_3 \vee \overline{x_4}$\newline
$c_3 = x_1 \vee x_2 \vee x_4$\newline
\textbf{Output}: \newline
True \newline
\textbf{Certificate}: \newline
$x_1= 1, x_2=0, x_3=0, x_4=0$

\subsubsection{Subset sum}
Given a set $S=\{x_1, x_2, ..., x_n\}$ of integers and an integer $t$, is there a non-empty subset $F \subseteq S$ such that its sum equals to $t$? That is: $\sum F=\sum_{x\ in F}x=t$
\newline
\newline
\textbf{Example}:
\newline
\textbf{Input}: \newline
$S=\{1,4,3,7,9\}$\newline
$t = 15$\newline
\textbf{Output}: \newline
True \newline
\textbf{Certificate}: \newline
$F=\{1,4,3,7\}$
\newpage
\subsection{Reduction}
We have to provide a function $f(p) \in P$ such that:
\[
\text{SUBSET-SUM}(f(p)) \iff \text{3SAT}(p)
\]
Because $p$ is input of 3SAT, $p$ is of the form $(U, G)$. Our function $f$ has to transform $p$ to a set of integers $S$ and an integer $t$ ($f(p)$ is input of Subset Sum) such that exists a subset $F \subseteq S$ and $\sum F=t$ if, and only if, all the clauses in $G$ are satisfiable with an assignment of the variables in $U$. In other words, if 3SAT($p$) is true, then SubsetSum(f(p)) has to be also true, and otherwise.

First, let's construct $f$:
\newline
\begin{itemize}
    \item We generate a set $S$ of numbers $a_i$ base 10 with digits $d_1,..., d_{|G|},d_{|G|+1},...,d_{|G|+|U|}$. This number $a_i$ will represent a truth assignation for one of the variables in $U$ and the clauses that the assignation satisfies. The first $|G|$ digits represent which of the clauses in $G$ are satisfied (1 if it's satisfied, 0 otherwise). The later $|U|$ digits indicates which variable has been assigned (for the variable $x_i$ the digit $d_{|G|+i}=1$, 0 otherwise. Regardless of the value of $x_i$). We generate a number for each truth and false assignation, this is $|S|=2|U|$.
    \item $t$ will be a number with the same structure as the $a_i$ numbers. The first $|G|$ digits will have value 3 (At least one has to be true for the clause to be true, but we can have at most 3). The next $|U|$ digits will have value of 1. As we want one assignation for every variable.
\end{itemize}

\textbf{For example:}
\newline
$U=\{x_1, x_2, x_3, x_4\}$\newline
$G=\{c_1, c_2, c_3\}$\newline
$p=(U, G)$\newline
\textbf{Where:}\newline
$c_1 = x_1 \vee x_2 \vee \overline{x_3}$\newline
$c_2 = \overline{x_2} \vee x_3 \vee \overline{x_4}$\newline
$c_3 = x_1 \vee x_2 \vee x_4$\newline
\textbf{Then $f(p)=(S, t)$ where}:\newline
$a_1 = 1011000$ (This is, the true assignation for variable $x_1$ (last 4 digits) satisfies clauses $c_1$ and $c_3$ (first 3 digits))\newline
$a'_1 = 0001000$ (The false assignation doesn't satisfy any clause)\newline
$a_2 = 1010100$\newline
$a'_2 = 0100100$\newline
$a_3 = 0100010$\newline
$a'_3 = 1000010$\newline
$a_4 = 0100001$\newline
$a'_4 = 0010001$\newline
$S=\{a_1, a'_1, a_2, a'_2, a_3, a'_3, a_4, a'_4 \}$\newline
$t=3331111$\newline
