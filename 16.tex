\section{Problema 16}
\subsection{Enunciat}
Consider the \textbf{Set Packing problem} as defined in problem 13 and its integer linear programming formulation on variables $x_1,...,x_n$ where variable $x_i$ indicates whether set $S_i$ is or is not in the packing. Consider the following algorithm that, first computes an optimal solution $x^{*}$ of the LP obtained by relaxing $x_i \in [0,1]$, and second performs the following rounding algorithm:
\begin{itemize}
    \item (1) Choose a set $S_i$ to be in the solution with probability $\frac{x^{*}_i}{6}$.\\
    \item (2) If an element $u \in U$ is covered by more than one set, remove all the sets in the solution that contain $u$.
\end{itemize}
\subsection{Solució}
Hem de demostrar que l'algorisme donat es una 12-aproximació de \textbf{Set Packing}.
\subsubsection{Demostració}
El valor òptim es $C=\sum_{\forall_i}x_ic(S_i)$ on $x_i$ és la solució de la formulació ILP donada al problema 13.
Donats $Sol_1$ com el resultat del algorisme després del pas 1 i $Sol_2$ el resultat després del pas 2 i $a_i$ una variable indicadora tal que:
\[
    a_{i} = 
\begin{cases}
    1,  & \text{si } S_i \in Sol_2\\
    0,  & \text{altrament}
\end{cases}
\]
Definim $C^{*}$ com el valor del profit a $Sol_2$:
\[
C^{*} = \sum_{\forall_i} a_i c(S_i)
\]
Per tant:
\[
E[C^{*}]=\sum_{\forall_i}P[a_i=1]c(S_i)
\]
\begin{eqnarray}
    P[a_i=1]=P[S_i \in Sol_1] \cap P[S_i \in Sol_2 | S_i \in Sol_1] \nonumber\\
    P[S_i \in Sol_1] = \frac{x_i^{*}}{6} \nonumber\\
    P[S_i \in Sol_2 | S_i \in Sol_1] = P[\bigcap_{S_j \cap S_i \neq \emptyset}S_j \notin Sol_1] = 1-P[\bigcup_{S_j \cap S_i \neq \emptyset} S_j \in Sol_1]\\
    P[\bigcup_{S_j \cap S_i \neq \emptyset}S_j \in Sol_1] \leq \sum_{S_j \cap S_i \neq \emptyset} P[S_j \in Sol_1] = \frac{1}{6}\sum_{S_j \cap S_i \neq \emptyset} x_j^{*}\leq \frac{1}{2}\\
    \Rightarrow 1-P[\bigcup_{S_j \cap S_i \neq \emptyset} S_i \in Sol_1] \geq \frac{1}{2}\nonumber\\
    \Rightarrow P[a_i=1] \geq \frac{x_i^{*}}{6}*\frac{1}{2}\geq \frac{x_i}{12}
\end{eqnarray}
\begin{itemize}
    \item (4) La probabilitat de passar el filtre del pas 2 es la probabilitat de que cap dels sets que comparteixen elements amb ell pertanyin a $Sol_1$, en altres paraules, la probabilitat complementaria a que algun dels que comparteixen elements pertanyi a $Sol_1$.
    \item (5) Com que cada set conté 3 elements, $\sum_{S_j \cap S_i \neq \emptyset}x_j^{*} \leq 3$, ja que només poden compartir algun dels 3 elements i les variables només poden arribar a 1 segons la formulació LP
    \item (6) Ja que LP$>$ILP
\end{itemize}
\qed