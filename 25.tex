\section{Problema 25}
\subsection{Enunciat}
Show that the Traveling Salesman problem parameterized by the number of cities belongs to FPT.
\subsection{Solució}
Per demostrar que un problema pertany a FPT, hem de donar un algorisme amb cost $O(f(k)*|x|^{O(1)})$.
Proposem el següent algorisme que soluciona la versió decisional del Traveling Salesman Problem:
\newline
\par
\textbf{TSP-DEC-BF($G$, $d$, $k$)}: On $G$ es un conjunt de tuples de mida $n$ i cada número a la posició $i$ representa la distància a la ciutat $i$ (representat amb $b_i$ bits), $d$ es un número que representa la distància màxima permesa (representat amb $b_d$ bits) i $k$ el màxim de ciutats. Genera tots els $\frac{(n-1)!}{2}$ cicles possibles\footnote{A un graf complet, qualsevol combinació de tots els nodes és un cicle hamiltonià ($n!$ combinacions). En les $n!$ combinacions trobem cada cicle repetit $n$ vegades (es comença el cicle des de cada vèrtex, d'aquí $(n-1)!$). A més, cada cicle està repetit un cop, ja que el trobem en les dues direccions (d'aquí la divisió per 2)} i retorna cert si es poden visitar tots els nodes de $G$ utilitzant una distància menor o igual a $d$ i utilitzant $k$ o menys nodes. Altrament retorna fals.
\newline
\newline
Podem veure que l'algorisme consistiria bàsicament en un bucle que realitza $\frac{(n-1)!}{2}$ iteracions i una comprovació. Cal tenir en compte el cost de realitzar operacions amb números grans $b>>k$. Per tant, el cost de l'algorisme resulta en $O(k!*b_m+b_d)$ on $b_m = max_{i=0}^{n}(b_i)$. En altres paraules, el cost de fer $k!$ operacions amb números de $b_m$ bits més el cost de fer una comprovació amb un número de $b_d$ bits. Finalment, veiem que $f(k) = k!$ ($k!$ és una funció de $k$) i $b_m, b_d < |x|$ (tant $b_m$ com $b_d$ són més petits que la mida de la entrada). Per tant, el cost és $O(f(k)*|x|^{O(1)})$ i queda demostrat que TSP parametritzat pel número de ciutats pertany a FPT.
\qed
